%%%%%%%%%%%%%%%%%%%%%%%%%%%%%%%%%%%%%%%%%%%%%%%%%%%
%% Document Class

\documentclass[review,11pt]{elsarticle}
%\documentclass[final,5p,times,twocolumn]{elsarticle}

\journal{Fire Safety Journal}

%%%%%%%%%%%%%%%%%%%%%%%%%%%%%%%%%%%%%%%%%%%%%%%%%%%
%% Bibliography

\bibliographystyle{elsarticle-num}
\biboptions{comma,square,numbers,sort&compress}

%%%%%%%%%%%%%%%%%%%%%%%%%%%%%%%%%%%%%%%%%%%%%%%%%%%
%% Packages

\usepackage{lineno}
\usepackage{booktabs}
\usepackage{graphicx}
\usepackage{amsmath}
\usepackage{amssymb}

%\usepackage{xr-hyper}
\usepackage[pdftex,
        colorlinks=true,
        urlcolor=linkblue,     % \href{...}{...} external (URL)
        citecolor=linkred,     % citation number colors
        linkcolor=linknavy,    % \ref{...} and \pageref{...}
        pdfproducer={pdflatex},
        pagebackref,
        pdfpagemode=UseNone,
        bookmarksopen=true,
        plainpages=false,
        verbose]{hyperref}

\usepackage{geometry}
\geometry{letterpaper, margin=1in}

\usepackage{lipsum}

\usepackage[nooneline]{subfigure}
%\subfigtopskip = 0cm
\subfigcapskip = -5mm
\subfigcapmargin = -3.5mm
%\subfigcaptopadj = 7cm
%\subfigbottomskip = 0cm
%\subfiglabelskip = -1cm

\usepackage{hyperref}
\hypersetup{colorlinks=true,linkcolor=blue}

\usepackage{caption}

\renewcommand*{\today}{July 11, 2018}

%%%%%%%%%%%%%%%%%%%%%%%%%%%%%%%%%%%%%%%%%%%%%%%%%%%

\begin{document}

\begin{frontmatter}

\title{Proceedings of the First Workshop Organized by the IAFSS Working Group on Measurement and Computation of Fire Phenomena (MaCFP)}


\author[label:Sandia]{A.~Brown}
\author[label:NIST]{M.~Bruns}
\author[label:UMD]{M.~Gollner}
\author[label:Sandia]{J.~Hewson}
\author[label:UMD]{A.~Marshall}
\author[label:NIST]{R.~McDermott}
\author[label:Ghent]{G.~Maragkos}
\author[label:Ghent]{B.~Merci}
\author[label:Poitiers]{T.~Rogaume}
\author[label:UMD]{S.~Stoliarov}
\author[label:UMD]{J.~Torero}
\author[label:UMD]{A.~Trouv\'{e}\corref{cor1}}
\ead{atrouve@umd.edu}
\author[label:FM]{Y.~Wang}
\author[label:Waterloo]{E.~Weckman}

\address[label:Sandia]{Fire Science and Technology Department, Sandia National Laboratories, Albuquerque,~NM~87185,~USA}
\address[label:NIST]{Fire Research Division, National Institute of Standards and Technology, Gaithersburg,~MD~20899,~USA}
\address[label:UMD]{Department of Fire Protection Engineering, University of Maryland, College~Park,~MD~20742,~USA}
\address[label:Ghent]{Department of Flow, Heat and Combustion Mechanics, Ghent University-UGent,~B-9000~Ghent,~Belgium}
\address[label:Poitiers]{Institut Pprime (UPR 3346 CNRS), Universit\'{e} de Poitiers, Isae-ENSMA,~86961~Futuroscope~Chasseneuil~Cedex,~France}
\address[label:FM]{FM Global, Research Division, Norwood,~MA~02062,~USA}
\address[label:Waterloo]{Department of Mechanical and Mechatronics Engineering, University of Waterloo, Waterloo,~Ontario,~N2L~3G1,~Canada}

\cortext[cor1]{Corresponding author}

\begin{abstract}

This paper provides a report of the discussions held at the first workshop on Measurement and Computation of Fire Phenomena (MaCFP) on June 10-11 2017. The first MaCFP workshop was both a technical meeting for the gas phase subgroup and a planning meeting for the condensed phase subgroup. The gas phase subgroup reported on a first suite of experimental-computational comparisons corresponding to an initial list of target experiments. The initial list of target experiments identifies a series of benchmark configurations with databases deemed suitable for validation of fire models  based on a Computational Fluid Dynamics approach. The simulations presented at the first MaCFP workshop feature fine grid resolution at the millimeter- or centimeter-scale: these simulations allow an evaluation of the performance of fire models under high-resolution conditions in which the impact of numerical errors is reduced and many of the discrepancies between experimental data and computational results may be attributed to modeling errors. The experimental-computational comparisons are archived on the MaCFP repository~\cite{MaCFP_repository}. Furthermore, the condensed phase subgroup presented a review of the main issues associated with measurements and modeling of pyrolysis phenomena. Overall, the first workshop provided an illustration of the potential of MaCFP in providing a response to the general need for greater levels of integration and coordination in fire research, and specifically to the particular needs of model validation.

\end{abstract}

\begin{keyword}
Buoyant plumes \sep Pool fires \sep Wall fires \sep Flame extinction \sep Fire modeling \sep Pyrolysis modeling \sep Large Eddy Simulation
\end{keyword}

\end{frontmatter}

\pagenumbering{arabic}
\setcounter{page}{1}
\linenumbers

% change "include" to "input" for final paper

\input{Introduction}

\section{Gas Phase Subgroup} \label{sec:GPS_session}

\input{CFD_VV_Review}

\input{Case_1_Turbulent_Plumes}

\input{Case_2_Gaseous_Pool_Fires}

\input{Case_3_Liquid_Pool_Fires}

\input{Case_4_Wall_Fires}

\input{Case_5_Flame_Extinction}

\input{Plans}

\section{Condensed Phase Subgroup} \label{sec:CPS_session}

% !TEX root = macfp_2017_gasphase.tex

\subsection{Objectives} \label{sec:CPS_session_1}

The production of combustible gases by burning materials is typically the rate-limiting process in the growth of fire.  A quantitative understanding of this process is therefore essential for advancing our ability to predict and mitigate fire development.  Unfortunately, measurement and modeling efforts carried out in this field by various research groups tend to be poorly coordinated.  Little agreement exists as to what constitutes best practices and standards in data collection and model development.  The purpose of the condensed phase subgroup of the MaCFP Working Group is to facilitate data and model sharing among researchers in order to improve predictions of thermal decomposition and pyrolysis in fire.  The work of this subgroup, in conjunction with the work of the gas phase subgroup, is expected to lead to fundamental progress in fire modeling.  It is envisioned that the two subgroups will collaborate to make quantitative predictions of the combined gas-solid phase processes that determine flame spread.  It is also envisioned that the two subgroups will hold joint workshops every two or three years.

The condensed phase subgroup shares the central objective of MaCFP ``to target fundamental progress in fire science and to advance predictive fire modeling."  The specific objectives of the subgroup will focus on the development, calibration, verification, and validation of predictive models of thermal decomposition and pyrolysis.  To this end, the subgroup plans to:

\begin{itemize}
\item Develop several alternative formats for experimental data sets that carry sufficient information to enable parameterization of pyrolysis models for a given material.
\item Develop a set of requirements for data set quality and completeness and organize a committee of experts that will review the submissions to the repository to ensure that they are compliant with these requirements.
\item Incorporate compliant data sets into the existing MaCFP data repository~\cite{MaCFP_repository}.
\item Create a database of pyrolysis property sets that are generated from the experimental data sets.  Each pyrolysis property set will be required to be accompanied by a demonstration of how well it captures the data on the basis of which it was calibrated and validated.
\item Develop a set of minimum requirements for numerical pyrolysis simulation codes.
\item Organize a discussion group focused on unresolved issues in pyrolysis modeling.
\end{itemize}

The scientific topics covered by the condensed phase subgroup will include:

\begin{itemize}
\item Kinetics and thermodynamics of the condensed phase decomposition reactions.
\item Properties and composition of gaseous pyrolyzates.
\item Heat and mass transfer in the condensed phase.
\item Physics and chemistry of the gas-condensed phase interface including the topics of oxidative pyrolysis and interactions with the surface flame.
\item Coupled thermal and mechanical behavior of pyrolyzing solids including intumescence and melt flow.
\end{itemize}



% !TEX root = macfp_2017_gasphase.tex

\subsection{Summary of the Planning Meeting} \label{sec:CPS_session_2}

As explained in section~\ref{sec:intro}, in addition to being a first technical meeting for the gas phase subgroup, the June-2017 MaCFP workshop served as a planning meeting for the condensed phase subgroup. The planning meeting featured an introductory presentation by the co-chairs of the condensed phase subgroup, followed by seven invited presentations and two periods for an open discussion. Hard copies of the presentations can be found in~\cite{MaCFP_wks_presentations}.

The introductory presentation~\cite{CPS:Bruns} presented an overview of the motivation, purpose, and goals of the condensed phase subgroup.  It was emphasized that fire phenomena can only be predicted with robust coupling between condensed and gas phase models.  Consequently, it will be necessary for the two subgroups of MaCFP to work closely in the planning and analysis of validation data as well as in subsequent model development.  Several of the challenges associated with condensed phase fire physics were mentioned.  Overcoming these challenges requires systematic verification and validation of condensed phase models.  Several concepts from validation and verification were reviewed including the so-called ``validation pyramid" as a heuristic for systematically validating complex models via sequential validation of various submodels.  The International Workshop on Measurement and Computation of Turbulent Nonpremixed Flames (known as the TNF workshop) was mentioned as a model for organizing the condensed phase subgroup's activities.  The reference to the TNF model led to a brief description of a proposed plan for the subgroup's work as presented in the White Paper~\cite{MaCFP_website} prepared by the co-chairs prior to the workshop.  The core of the proposal is to facilitate communication between experimentalists and modelers by providing web-based management of four elements:  (1) experimental data; (2) numerical models; (3) parameter sets and associated comparisons between model predictions and experimental data; and (4) a discussion forum.  For each of these elements, the presentation provided a brief explanation as well as some proposed constraints.  First, the experimental data should initially focus on scenarios in which flaming is not present so that condensed phase physics may be isolated.  This data will need to follow some requirements for formatting and review.  Second, the numerical models should be open source, well-documented, and include at least heat transfer and decomposition reaction kinetics.  Third, the parameter sets and comparisons should be complete and the link between the underlying data and the parameter values should be specified.  The Fire Dynamics Simulator (FDS) Validation Guide~\cite{FDS_Validation_Guide} was mentioned as a good example of such comparisons.  Finally, several topics such as missing experimental data, needed model developments, and computational challenges were suggested for the discussion forum.  It was noted that a successful discussion forum will require sustained community participation.

The first invited presentation~\cite{CPS:Leventon} began with an overview of different condensed phase models, from early heat-transfer-based analytical models for ignition up to modern computational pyrolysis solvers such as FDS~\cite{FDS_Math_Guide}, Gpyro~\cite{Lautenberger:2014} and ThermaKin~\cite{Stoliarov:2014}.  These modern computational models rely on a relatively large number of material properties used to characterize pyrolysis behavior.  A list of common material properties used in computational pyrolysis models is provided in Table~\ref{tab:table1}.  Identifying values for these many parameters presents a challenge especially as the values can change significantly as a material heats and decomposes.  The remainder of the presentation focused on describing a procedure for determining the kinetic and thermodynamic properties of materials developed at the University of Maryland~\cite{Stoliarov:2015}.  This procedure relies on data from three milligram-scale experiments:  (1) thermogravimetric analysis (TGA) for decomposition reaction kinetics; (2) differential scanning calorimetry (DSC) for heat capacities and heats of decomposition reactions; and (3) microscale combustion calorimetry (MCC) for heats of combustion of gaseous pyrolyzates.  The presentation described these experiments and procedures for extracting material properties from the appropriate data.  Throughout the discussion, data for poly(butylene terephthalate) (PBT) was used as an example. 

\begin{table}
\caption{Material properties required by state-of-the-art computational pyrolysis models.}
\centering
\footnotesize
\makebox[\textwidth]{
\begin{tabular}{p{1.75in}p{1.95in}p{1.75in}}
\toprule
 Kinetic & Thermodynamic & Transport \\
\midrule
\midrule
 Pre-exponential factors     & Specific heat capacities                 & Thermal conductivities \\
 Activation energies            & Heats of decomposition reactions & Emissivities \\
 Stoichiometric coefficients & Heats of combustion                     & Absorption coefficients \\
                                            &                                                      & Mass diffusivities \\
\bottomrule
\end{tabular}
}
\label{tab:table1}
\end{table}

The second invited presentation~\cite{CPS:Brown} discussed current work on validating models of solid reacting materials.  Surface temperature measurements using thermophosphors are being explored, and datasets for solid reactive materials are being generated.  The focus at Sandia National Laboratories (Sandia) is on the high heat flux regime.  A number of test facilities are available for high heat flux ignition experiments including the Sandia solar furnace which can provide a heat flux of 5 MW/m$^2$.  In addition to experimental work, Sandia is developing a code for fire modeling (Fuego~\cite{SIERRA/Fuego}) and a code for reacting solid materials (Aria).

The relationship between gas and condensed phase physics was discussed in the third invited presentation~\cite{CPS:Richard}.  The large number of physical processes occurring at the solid-gas interface were enumerated.  It was emphasized that ignition and flame spread are significantly influenced by the details of transport and chemistry occurring at the interface.  A review of boundary layer theory was provided as well as a discussion of the reacting boundary layer theory of Emmons.  The presentation concluded by highlighting the need to develop better models of turbulence, heat transfer, and combustion in the near-wall region of the boundary layer to account for chemical and blowing effects corresponding to pyrolysis.

The fourth invited presentation~\cite{CPS:Hostikka} provided a discussion of the solid model implemented in FDS~\cite{FDS}.  The presentation began by presenting a schematic of the physical processes involved in burning materials and emphasized the multi-scale nature of the problem.  The governing equations for condensed phase species and energy conservation as well as the pore gas conservation equations for species, energy and momentum, were presented.  The need to limit the model to include only the important physics was noted.  Following on from this point, the presentation listed the major assumptions made by the FDS solid model.  Specifically, the FDS solid model assumes one-dimensional transport, no mass accumulation (and therefore instantaneous mass transfer), thermal equilibrium between gases and solids, and assumes that a heat of reaction may be used to account for the energy contribution of the decomposition reactions.  The FDS solid model is regularly verified (in terms of heat conduction, radiation, mass conservation, and reaction rate) and validated (in terms of mass loss rate and heat release rate for burning polymer slabs).  Several special topics for future pyrolysis model development were mentioned including spectral radiation, shrinking and swelling, pressure build-up, multi-dimensional effects, and solid mechanical considerations associated with fracturing of char.  The presentation concluded by suggesting that these special topics might be appropriate for further exploration by the MaCFP condensed phase subgroup.

The challenge of coupling condensed and solid phase models was explored in the fifth invited presentation~\cite{CPS:Wang}.  In contrast to the multi-experiment approach developed at the University of Maryland~\cite{CPS:Leventon}, FM Global calibrates all material properties using data from a single experiment, namely the fire propagation apparatus (FPA).  A one-dimensional model with a single-step Arrhenius reaction is then fit to the FPA data using optimization with the shuffled complex evolution (SCE) algorithm.  The resultant pyrolysis properties are then used as inputs in a CFD fire model (FireFOAM~\cite{FireFOAM}) to simulate full-scale fire scenarios.  Validation of this approach has been performed for several additional FPA scenarios.  An important application of fire models for FM Global is understanding fire spread in warehouse rack storage.  FireFOAM has been used to predict heat release rate in 3-tier, 5-tier, and 7-tier rack storage of cardboard boxes using properties obtained by the FPA/SCE material property calibration procedure.  The presentation also discussed applications involving boxes of plastic cups and large rolls of paper.  The paper rolls present a unique challenge in that delamination of outer layers of paper had to be accounted for.  Several lessons were provided in conclusion.  First, coupling of gas and condensed phase models is necessary for real-world problems, and the appropriate level of model complexity is determined by the problem.  Second, validation needs to occur both for the decoupled and coupled models at multiple scales.

The sixth invited presentation~\cite{CPS:Rein} discussed recent work on assessing the appropriate level of model complexity.  Beginning with a clear statement of the goal to ``up-scale" from fundamental physics and chemistry to real fire behavior, the presentation laid out some of the many challenges in the path of achieving that goal.  One of those challenges is choosing the appropriate level of model complexity.  The number of parameters in pyrolysis models can vary from just a few to over 30 for some of the more detailed models in existence.  The problem of complexity is one of finding the minimum number of parameters required to attain an acceptable level of error.  In the recent work presented in Refs.~\cite{Rein:2013,Rein:2015}, this problem has been addressed by systematically decreasing model complexity used to predict the pyrolysis of a vertical slab of PMMA exposed to varying levels of heat flux and oxygen.  It was found that it is not helpful to increase the complexity of the chemical model unless a sufficiently complex model of heat transfer is used.  Furthermore, additional complexity corresponds to increased uncertainty, and so complex models should only be used in the presence of sufficient, high quality data.

Finally, the seventh invited presentation~\cite{CPS:Lautenberger} provided an overview of pyrolysis modeling with Gpyro.  Gpyro is an open-source three-dimensional pyrolysis model with user-specified complexity.  Additionally, Gpyro may be coupled to FDS with some limitations ($e.g.$ Cartesian geometries, no shrinkage or swelling, and no burn-away).  Current work involves coupling to ABAQUS for mechanical calculations.  A critical part of any pyrolysis solver is the material property models allowed.  Gpyro treats material properties as weighted sums of species properties with a power law dependence on temperature.  Additionally, permeability and thermal conductivity may be anisotropic which can be important for materials such as wood.  After going through the form of the conservation equations, the presentation provided some details on the numerical schemes employed by Gpyro.  The time-stepping is fully implicit to ensure stability of the solution, and an alternating direction tri-diagonal matrix algorithm is utilized for speed.  Several verification cases have been carried out including for a sphere with internal heat generation.  As an example of the power of detailed pyrolysis modeling, the presentation gave the example of wood pyrolysis at both small and large external heat fluxes, with the ``fast" case producing significantly more tar as compared to the ``slow" case.

The invited presentations served to establish a foundation for the two periods of open discussion that were scheduled in the workshop.  These open discussion periods were crucial for fielding input from the research community at large.  The issue of heating rate in small scale experiments was discussed.  Some believe that the heating rates used in small-scale tests should emulate realistic fire heating rates, but it was noted that chemical reaction kinetics generally depend on temperature (not heating rate) and increasing the heating rate does not significantly change the temperature range over which solid decomposition reactions take place.  Furthermore, high heating rates can lead to temperature and species concentration gradients which prevent meaningful interpretation of results of small-scale tests such as TGA.  Several participants suggested that TGA should be coupled with gas analysis (e.g., FTIR) as this information could be important for the gas phase physics.  Similarly, the impact of oxygen concentration should be explored further in order to model the transitional regimes of ignition, spread, and extinction in contrast to steady burning.  For all small-scale tests, it was suggested that it should be necessary to precisely describe the range of validity of the parameters as a consequence of how they are determined.  Much of the open discussion time was devoted to identifying appropriate validation data sets.  Some participants suggested that it is important to have large-scale data (such as the FM Global parallel panel test or the standard room corner tests) early on in order to better guide subsequent model development and experimentation.  This would not negate the necessity of small-scale tests for model parameterization or validation of sub-models, but inversely, this illustrates the pertinence of the up-scaling approach. Another issue that arose is the appropriate selection of test materials.  A balance should be struck between simplicity for modeling purposes and real-world application.




% !TEX root = macfp_2017_gasphase.tex

\subsection{Future Plans} \label{sec:CPS_session_3}

Future steps include the development of a digital archive dedicated to the condensed phase subgroup, possibly using the same platform as the gas phase subgroup~\cite{MaCFP_repository}.  In parallel, the standards for the experimental data sets will be established.  The following standards are proposed:

\begin{itemize}
\item Each studied material must have clearly defined chemical composition.  The material's physical attributes, such as color, initial density and thickness, geometry of reinforcement (in the case of structural composites), must be provided.  The material should be readily available, preferably, from multiple distributors.
\item The material should be conditioned prior to all experiments in a well-defined atmosphere with these conditions specified.  For hydrophilic materials, the initial moisture content should be reported.
\item The experiments used to determine properties may consist of milligram-scale and/or gram-scale tests.  Milligram-scale tests (such as TGA) are expected to be conducted under thermally thin conditions, $i.e.$ conditions for which the sample temperature is spatially uniform and resolved in time.  Gram-scale tests (such as FPA gasification experiments~\cite{Chaos:2011}) are expected to be conducted under non-thermally thin conditions, $i.e.$ conditions for which transport properties have significant impact on the measured quantities.  Gram-scale tests must have well-defined thermal boundary conditions, more specifically, heat fluxes incident on all sample surfaces should be specified as a function of surface temperature.  If the material sample is mounted onto thermal insulation, the properties of this insulation must be provided.  The composition of the gaseous environment inside all test apparatus must be defined. For all tests, the size and mass of the samples must be fully specified.
\item Each data set may contain either milligram- and gram-scale test results or, alternatively, only gram-scale test results.  In the latter case, the results from multiple experiments performed at a range of heating conditions must be reported and include time-resolved sample mass as well as sample temperature measurements (at the surface and/or at an in-depth location). The conditions of the tests (for example heating rate, temperature range, percentage of oxygen in the atmosphere) must be defined.
\item Heat of combustions of gaseous pyrolyzate produced by the material must be measured using a Cone Calorimeter, FPA, or Microscale Combustion Calorimeter.
\item Additional data, including chemical composition of the gaseous pyrolyzate, and thermal conductivity, emissivity, radiation absorption coefficient and mass diffusivity of the solid, are desirable but not required.
\item All experimental data must contain information on their uncertainties.
\end{itemize}

Multiple experimental data sets for the same material will be allowed into the repository, provided that each of them satisfies all established requirements.  One key requirement for each pyrolysis property set, which will be generated from the experimental datasets, is a demonstration of how these property values capture all data in at least one experimental dataset, $e.g.$ if the dataset contains the results of controlled-atmosphere cone calorimetry experiments and TGA, the developer of the pyrolysis property set will be required to produce predictions of both of these experiments and provide input files that were used to generate these predictions.  Quantitative criteria will be developed to characterize the quality of each prediction. 

It is proposed that, initially, experimental data sets will be developed for relatively simple materials that are isotropic in nature and do not exhibit complex mechanical behavior such as melt flow, delamination or intumescence.  Examples of such materials include cast poly(methyl methacrylate) and high-impact polystyrene.  Demonstrating that the pyrolysis property sets can be used to successfully predict compartment-scale fire growth will be a longer term goal of this effort.  One potential target geometry for the full-scale experiments is upward and lateral flame spread in a flammable corner, which is realized in several flammability standards~\cite{NFPA,EN,ISO}.  It is proposed that well-instrumented versions of these standard experiments be carried out to serve as a modeling target for comprehensive gas and condensed phase models of fire growth.

In closing, the co-chairs of the condensed phase subgroup of the MaCFP Working Group have now started discussions for the organization of a second workshop. Interested individuals/organizations are encouraged to contact the co-chairs~\cite{MaCFP_website} in order to participate in preparations for the workshop and in the construction of the digital archive described above.


\section*{Acknowledgments} \label{sec:ack}

The authors would like to gratefully acknowledge the endorsement  and support of MaCFP by IAFSS. The authors would also like to acknowledge all the researchers who contributed to the development of the experimental databases that were used in the present validation work, in particular the authors of Refs.~\cite{Case1_EXP,Case2a_EXP,Case2b_EXP_CH4,Case2b_EXP_H2,Case3_EXP_1,Case3_EXP_2,Case4_EXP_1,Case4_EXP_2,Case5_EXP_1,Case5_EXP_2,Case5_EXP_3}. Finally, the authors would like to gratefully acknowledge the contributions of all the researchers who contributed to the development of the numerical databases for the first MaCFP workshop, in particular the authors of Refs.~\cite{Case1_SIM_IRSN,Case1_SIM_NIST,Case1_SIM_UGent,Case2a_SIM_FMG,Case2a_SIM_UGent,Case2a_SIM_IRSN,Case2a_SIM_NIST,Case2b_SIM_UGent,Case2b_SIM_NIST,Case2b_SIM_SNL,Case2b_SIM_UCantabria,Case3_SIM_UGent,Case3_SIM_UMD,Case3_SIM_VTT,Case4_SIM_NIST,Case4_SIM_FMG,Case5_SIM_FMG,Case5_SIM_NIST,Case5_SIM_UMD}.

\bibliography{MaCFP-1_Refs}

\end{document}




