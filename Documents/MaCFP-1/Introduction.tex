% !TEX root = macfp_2017_gasphase.tex

\section{Introduction} \label{sec:intro}

A new initiative, endorsed and supported by the International Association for Fire Safety Science (IAFSS)~\cite{IAFSS_website}, has been launched: the ``{\it IAFSS Working Group on Measurement and Computation of Fire Phenomena}" (or the MaCFP Working Group)~\cite{MaCFP_website}. The general objective of the MaCFP Working Group is to establish a structured effort in the fire research community in order to make significant and systematic progress in fire modeling through a fundamental understanding of fire phenomena. This is to be achieved as a joint effort between experimentalists and modelers on the general topic of the experimental validation of fire models based on a Computational Fluid Dynamics (CFD) approach. The MaCFP Working Group is intended as an open, community-wide, international collaboration between fire scientists. It is also intended to become a regular series of workshops, with workshops held every two or three years. The first workshop organized by the MaCFP Working Group was held as a pre-event to the 12th IAFSS Symposium in Lund, Sweden, on June 10-11 2017~\cite{MaCFP_wks_presentations}. This paper presents a summary of the discussions and outcomes of the first MaCFP workshop.

The content and format of the first MaCFP workshop had been previously decided during a planning meeting in 2015. The planning meeting had produced a list of target experiments with databases deemed suitable for validation of CFD-based fire models. The intent was to make sure that the first workshop would go beyond the level of general discussions and would include presentations of a first suite of experimental-computational comparisons corresponding to an initial list of relevant experiments. The list of target experiments and a call for participation in the first workshop were broadly advertised to the international fire research community through letters to the editors of {\it Fire Safety Journal}~\cite{MaCFP_FSJ} and {\it Fire Technology}~\cite{MaCFP_FT} as well as through emails to the IAFSS membership.

While early discussions of the MaCFP Working Group had focused on gas phase phenomena (primarily flow and combustion phenomena), discussions were started in 2016 to expand the scope of MaCFP to include a subgroup dedicated to the modeling of pyrolysis phenomena. This led to a re-structuring of MaCFP into two subgroups: the (original) ``{\it gas phase subgroup}" and the (new) ``{\it condensed phase subgroup}". Thus, in addition to being a first technical meeting for the gas phase subgroup, the June-2017 MaCFP workshop also served as a planning meeting for the condensed phase subgroup. The planning meeting portion of the workshop included a review of the main issues associated with pyrolysis measurements and modeling for fire applications and a discussion of future priorities for the condensed phase subgroup.

The technical meeting portion of the workshop provided a first demonstration of current activities of the MaCFP Working Group as well as an illustration of their potential impact. The initial list of target experiments identified by the gas phase subgroup corresponds to basic configurations (also called building blocks) featuring carefully-controlled conditions and quality instrumentation and diagnostics. They also correspond to experiments with open, easily-accessible databases. In what is considered as a first intermediate step, the list has a limited scope and only includes simple turbulent buoyant plumes and simple flames (in most cases, the flames are non-sooting or only weakly-sooting), supplied with gaseous or liquid fuel, and featuring open burn conditions; the case of strongly sooting and smoking flames, fueled by solid flammable materials, and featuring compartment effects is outside the scope of the first MaCFP workshop and will be considered in future editions.

The initial list of MaCFP target experiments includes five categories:
\begin{itemize}
\item (Case 1) Turbulent buoyant plumes: this category corresponds to open plumes and is represented by a helium plume experiment conducted at Sandia National Laboratories (Sandia)~\cite{Case1_EXP}.
\item (Case 2) Turbulent pool fires with gaseous fuel: this category corresponds to open flames with a prescribed fuel flow rate and is represented by a series of natural gas flame experiments conducted at the National Institute of Standards and Technology (NIST)~\cite{Case2a_EXP} (and referred to in the following as the NIST McCaffrey natural gas flame experiment) and by a series of methane and hydrogen fire experiments conducted at Sandia National Laboratories (Sandia)~\cite{Case2b_EXP_CH4,Case2b_EXP_H2}.
\item (Case 3) Turbulent pool fires with liquid fuel: this category corresponds to open flames with a thermal-feedback-driven fuel flow rate and is represented by a methanol pool fire experiment conducted at the University of Waterloo (UW)~\cite{Case3_EXP_1,Case3_EXP_2}.
\item (Case 4) Turbulent wall fires: this category corresponds to boundary layer flames with a prescribed fuel flow rate and is represented by a series of vertical wall flame experiments, fueled by methane, ethane, ethylene or propylene, and conducted at FM Global~\cite{Case4_EXP_1,Case4_EXP_2}.
\item (Case 5) Flame extinction: this category corresponds to flames driven to extinction conditions and is represented by a series of methane and propane line flame experiments conducted at the University of Maryland (UMD)~\cite{Case5_EXP_1,Case5_EXP_2,Case5_EXP_3}.
\end{itemize}

Note that the experimental databases corresponding to Cases 1-5 are hosted on the MaCFP repository~\cite{MaCFP_repository} with open access so that the data are available to the fire research community as reference data for future experimental and/or computational studies.

Seven groups submitted computational results for comparisons with experimental data and for discussions at the first MaCFP workshop: FM Global (USA); Ghent University (UGent, Belgium); the Institut de Radioprotection et de S\^uret\'e Nucl\'eaire (IRSN, France); the National Institute of Standards and Technology (NIST, USA) teamed up with the VTT Technical Research Centre of Finland (VTT, Finland); Sandia National Laboratories (SNL, USA); University of Cantabria (UCantabria, Spain); and University of Maryland (UMD, USA). These groups used one of the following four CFD solvers:

\begin{itemize}
\item FDS (Fire Dynamics Simulator) developed by NIST in collaboration with VTT~\cite{FDS};
\item FireFOAM based on OpenFOAM~\cite{OpenFOAM} and developed by FM Global~\cite{FireFOAM};
\item ISIS developed by IRSN~\cite{ISIS};
\item SIERRA/Fuego developed by SNL~\cite{SIERRA/Fuego}.
\end{itemize}

These solvers are representative of current fire modeling capabilities available for research-level and/or engineering-level projects.

The paper is structured as follows. Section~\ref{sec:GPS_session} describes the main outcomes of the technical meeting held by the gas phase subgroup of the MaCFP Working Group. Section~\ref{sec:cfd_review} gives a brief description of the different concepts used in quality control of CFD models and a review of the computational challenges found in model validation (the focus of MaCFP). Sections~\ref{sec:buoyant_plumes}-\ref{sec:flame_extinction} present a summary of the experimental-computational comparisons performed for Cases 1-5, respectively. Section~\ref{sec:plans} presents a conclusion and a description of future plans for the gas phase subgroup. Section~\ref{sec:CPS_session} presents a review of the discussions held during the planning meeting of the condensed phase subgroup of the MaCFP Working Group. Section~\ref{sec:CPS_session_1} gives a brief description of the objectives of the subgroup. Section~\ref{sec:CPS_session_2} presents a summary of the invited presentations and follow-up discussion that took place at the workshop. Section~\ref{sec:CPS_session_3} presents a conclusion and a description of future plans for the condensed phase subgroup.




